\section{Situational Awareness Systems}

\subsection{General Situational Awareness Platforms}
Situation Awareness (SA) platforms constitute a core architectural component of modern tactical communication and disaster response systems, designed to enhance operators' cognitive capabilities in perceiving, comprehending, and projecting dynamic environmental conditions. The theoretical foundation primarily derives from Endsley's \cite{Endsley1995TowardAT} seminal three-level model, which delineates SA formation through hierarchical cognitive processes: perception of critical environmental cues, comprehension of their task-relevant significance, and projection of future situational trajectories. This model emphasizes the tight coupling between cognitive mechanisms and system functionalities, providing a structured framework for SA system design and evaluation \cite{DAniello2023FuzzyLF}.

At the implementation level, SA platforms must orchestrate heterogeneous data streams from distributed sensor networks, geographic information systems, real-time communication channels, and personnel status indicators to synthesize a unified, interactive operational picture that facilitates cross-agency coordination and time-critical decision-making. 

In disaster response contexts, for instance, the integration of UAV-based remote sensing, social media analytics, and on-site sensor networks substantially enhances real-time situational comprehension regarding disaster scope, resource distribution, and population dynamics \cite{Kedia2022TechnologiesES}. In military and security domains, uncertainty reasoning techniques such as fuzzy logic are extensively employed to mitigate the effects of noisy and incomplete information, thereby improving system robustness under complex operational conditions \cite{DAniello2023FuzzyLF}.


A predominant focus in the design of current SA platforms has been on achieving robust and real-time situation perception and comprehension. This design priority addresses the most immediate needs for decision-support by providing a faithful representation of the current operational state \cite{DAniello2023FuzzyLF}. Consequently, the capability for situation projection remains a recognized and open challenge within the field. Future advancements are anticipated to incorporate predictive modeling techniques, such as fuzzy cognitive maps and neuro-fuzzy systems, to evolve from descriptive to prescriptive analytics. 

Furthermore, operational hurdles like communication disruptions, data heterogeneity, and interoperability gaps, as noted in disaster response scenarios \cite{Kedia2022TechnologiesES}, must be overcome to fully realize such advanced capabilities. Consequently, effective projection capabilities are fundamentally dependent on robust perception and comprehension mechanisms. Given the current limitations in these foundational stages, strengthening the lower-level SA functions represents a critical prerequisite for advancing predictive capabilities in operational systems.


\subsection{Commercial and Military Tactical Situational Awareness Platforms}

Whereas the previous subsection examined general SA concepts and algorithmic challenges, the present work focuses on \emph{operational} systems that implement such capabilities in real-world military and civil protection environments. In practice, situational awareness in the field is provided not by abstract algorithms alone but through integrated platforms that combine geospatial Common Operating Pictures (COPs), distributed sensing, and tactical communication backends capable of functioning under degraded network conditions.

Among the broad landscape of SA solutions, two categories are of particular relevance to this thesis. The first comprises nation-scale, multi-domain platforms—such as Ukraine’s Delta system—that demonstrate the upper bound of modern digital command-and-control capabilities but rely on substantial communication infrastructure. The second comprises modular field-deployable platforms—most notably the Android Team Awareness Kit (ATAK)—which provide map-centric situational awareness to tactical units and can operate atop heterogeneous communication layers, including MANET radios. 

These platforms illustrate a consistent architectural pattern: while their software capabilities for perception and comprehension are mature, their operational effectiveness is tightly coupled to the availability of high-bandwidth, power-intensive communication substrates. As a consequence, existing systems offer limited support for ultra-low-power, infrastructure-independent last-mile connectivity required by lightweight edge devices in scenarios of complete infrastructure collapse. The remainder of this subsection examines Delta and ATAK as representative platforms, and motivates the LoRa-based communication extension developed in this thesis.


\subsubsection{Delta}

Delta, developed and continuously refined by Ukraine's Defense Innovation and Development Center since 2016, constitutes a nation-scale, multi-domain tactical situational awareness platform designed to support cross-service and inter-agency coordination across land, maritime, aerial, cyber, and space operational theaters\cite{Putrenko2024}. Extensively deployed throughout the Russo-Ukrainian War, Delta has been empirically validated as a central component of Ukraine's digitalized command-and-control infrastructure\cite{Putrenko2024}.

Architecturally, Delta adopts a cloud-native and lightweight design compatible with heterogeneous client devices (smartphones, tablets, laptops)\cite{Putrenko2024}. It performs comprehensive multi-source intelligence fusion by integrating satellite imagery, UAV video feeds, radar observations, SIGINT\footnote{signals intelligence }, ELINT\footnote{electronic intelligence }, and forward observer reports into a unified COP\footnote{Common Operating Picture}\cite{Putrenko2024}. This automated yet human-supervised fusion pipeline exemplifies a mature cyber-physical-social situational awareness paradigm capable of assimilating both sensor-derived information and crowd-sourced civilian reports.

Despite its operational effectiveness, Delta's communication architecture remains structurally dependent on high-bandwidth infrastructure, including Starlink satellite networks, military IP backbones, and commercial telecommunications links\cite{Putrenko2024}. While such connectivity enables large-scale and high-temporal-resolution information synthesis in satellite-enabled environments, the reliance on robust infrastructure significantly restricts the platform's applicability for edge nodes operating under conditions of complete infrastructure collapse.

\subsubsection{ATAK}

In contrast to Delta's nation-scale focus, the Android Team Awareness Kit (ATAK) represents a modular, device-centric tactical SA platform that is directly relevant to this thesis as the target environment for the proposed communication extension. ATAK provides a map-centric mobile geospatial collaboration environment and is designed to run on standard Android devices, thereby enabling tactical units to access situational information at the edge of the network.

Originally developed by the United States Department of Defense to fulfill military tactical requirements, the platform has subsequently garnered increasing popularity within civilian sectors\cite{ATAK}. Its widespread deployment across military units, special police forces, search and rescue teams, and emergency management agencies in the United States exemplifies its utility. In Germany, smartphone-based applications designed for emergency alerting have become increasingly vital tools for first responders, including personnel from fire departments, police forces, and various rescue organizations\cite{Linde2022EmergencyAT}. 

This proliferation aligns with Germany's decentralized model of emergency and disaster management. Within this framework, responsibilities are collectively shared across federal, state, and local governmental tiers, fostering collaborative participation between official agencies and non-governmental organizations\cite{Domres2000TheGA}. In this context, the functionalities offered by ATAK are highly congruent with the German emergency response system's demand for adaptable and efficient information processing instruments. This synergy indicates a substantial potential for its broader application within the German operational landscape.

\paragraph{Tactical Communication Backends for ATAK: The Role of MANET Radios}

In operational deployments, ATAK does not function as an isolated application but relies on high-performance tactical communication systems as its network substrate. As outlined in Section 1.3.2, MANETs form one of the dominant communication backbones for mobile tactical teams. One of the most prominent commercial implementations is Persistent Systems' fifth-generation MANET radio, the MPU5, which has been widely adopted by the U.S. Army, Marine Corps, law-enforcement agencies, and search-and-rescue organizations. The MPU5 provides a self-forming, self-healing multi-hop networking architecture capable of maintaining connectivity in infrastructure-denied environments.

The system employs the proprietary Wave Relay® protocol to enable distributed routing and low-latency communication across networks exceeding 250 mobile nodes. Its physical layer implements 2×2 or 3×3 MIMO configurations with advanced interference mitigation, supporting operation across 2.4 GHz, 5 GHz, L-band, and S-band frequencies. The MPU5 can simultaneously transport diverse mission-critical data streams—including real-time video from UAVs or body-worn cameras, push-to-talk voice, and IP-based tactical applications—thereby forming a unified communication ecosystem interconnecting unmanned systems, ground robots, and sensor arrays.

Native compatibility with ATAK and WinTAK further enables it to serve as the primary transport for blue-force tracking, collaborative map synchronization, and geospatial event distribution. Field performance in urban canyons, dense forests, subterranean environments, and smoke-obscured fire grounds has established the MPU5 as a standard backbone for operational ATAK deployments.

However, this capability comes at a fundamental cost: the MPU5's high energy consumption, substantial form factor, and operational expense preclude its deployment on lightweight, battery-constrained edge devices that must sustain multi-day or multi-week autonomous operation in conditions of complete infrastructure loss. Similar to the previously examined Delta platform, MPU5-based MANETs effectively solve theater-level tactical communications but leave unaddressed the requirement for ultra-low-power, infrastructure-independent last-mile connectivity—precisely the capability gap motivating the LoRa-based extension developed in this thesis.

\paragraph{Map-Centric User Interface and Collaboration Features}

To accommodate the complex and dynamic demands of tactical communication, ATAK's architectural design incorporates a highly modular approach, supporting the rapid integration of multi-source information and communication interfaces through its plugin mechanism.  Its user interface is centered around a map-based design, supporting the integration of multiple online map sources through the MOBAC\footnote{Mobile Atlas Creator} format. The platform also offers offline caching capabilities, allowing users to access high-resolution geospatial data even in disconnected environments. The overlay feature enables the display of static images, such as heat maps or scanned paper maps, directly on top of the digital map view, further enhancing tactical planning and situational awareness. 

In addition, ATAK provides a comprehensive set of annotation and collaboration tools: users can place standardized icons, draw, and annotate on the map, and share these markings in real time via team-wide broadcast mechanisms. 

\paragraph{Communication Agnosticism and Limitations}
The inherent adaptability of the platform's communication layer allows it to flexibly accommodate heterogeneous physical networks, including LTE, Wi-Fi and satellite links. In extreme scenarios where central routers or trusted nodes are absent, ATAK’s built-in GeoChat module establishes a logical broadcast domain using UDP multicast, enabling devices to synchronize information directly in a peer-to-peer manner. Because this mechanism operates entirely within the local network domain and does not require any external infrastructure, GeoChat can be rapidly deployed and used even in severely resource-constrained environments. 

However, a limitation of this module is its lack of message persistence. If a recipient is offline when a message is sent, the content won't be delivered, and there's no mechanism for retransmission. To address the need for persistent communication, the platform incorporates the TAKChat plugin, which is built on the XMPP protocol. This module provides full chat persistence, supporting cross-platform message delivery, server-side storage, and chat room permission management.

All data is ultimately encapsulated within the Cursor-on-Target (CoT) protocol, a lightweight XML-based structure developed by MITRE\cite{Jacoby2014AfterAR}. This protocol features a clear event model and semantic hierarchy. Each CoT message is composed of an \texttt{<event>} element, which includes fields such as a unique identifier (\texttt{uid}), event type (\texttt{type}), time information (\texttt{start}, \texttt{stale}, \texttt{time}), generation method (\texttt{how}), geographical coordinates (\texttt{<point>}), and extended details (\texttt{<detail>}). The system utilizes the \texttt{\textless detail\textgreater} node to flexibly carry extended data, such as \texttt{\_\_chat} for message bodies, \texttt{chatgrp} for group identifiers, \texttt{link} for associating entity relationships,  and \\\texttt{\_\_serverdestination} for explicitly specifying the message's target address and transmission protocol (e.g., \texttt{IP:PORT:PROTOCOL}). This protocol format is lightweight and semantically clear, supporting various information types including event types, geographical locations, and message content, thereby forming the foundation for inter-plugin collaboration and system compatibility.

\subsection{ATAK's Plugin Architecture for Customization}
The plugin mechanism of ATAK is one of its most distinctive architectural features. Plugins exist as independent \texttt{APK} (Android Package Kit) files and do not rely on a whitelist from the main application. Instead, the main application automatically identifies, loads, and registers them at runtime. Plugins extend modular functionality based on predefined TAK APIs, gaining direct access to core functional modules such as map rendering, message distribution, databases, location information, and contact lists.

Typical class structures within this architecture include:

\begin{itemize}
  \item \textbf{PluginLifecycle}: This component functions as the plugin's life cycle manager, analogous to Android's \texttt{Application} class, and is responsible for initialization and configuration hooks.

  \item \textbf{DropDownMapComponent}: This acts as the main controller, registering user interface logic, event listeners, and task scheduling components. It is functionally similar to an Android \texttt{Activity}.

  \item \textbf{DropDownReceiver}: This is primarily responsible for UI display, typically bound to a slide-out panel, and supports loading custom XML layouts.
\end{itemize}

The method DropDownReceiver.showDropDown() is responsible for presenting user interface elements in response to user actions such as button clicks, map interactions, or background events.


\vspace{0.5em}
Developing ATAK plugins requires Android Studio, with a minimum supported Android version of 5.0 (\texttt{API 21}) or higher. It is recommended to use Java 11, Gradle Plugin version \texttt{4.2.2}, and main project Gradle version \texttt{6.9.1} to ensure compatibility with the existing \texttt{ATAK-CIV} SDK. Plugins must possess a signing key for system validation, and their deployment path should be located within the \texttt{atak-civ/plugins} directory, maintaining consistent directory hierarchy with components such as \texttt{main.jar} and \texttt{atak.apk} within the SDK to ensure automatic loading.


\subsection{External Integration Approaches}

Beyond its internal plugin architecture, ATAK can also be connected to external software systems via standalone integration adapters that exchange data with the platform over IP-based transport using standardized CoT messages. Such designs are common in robotics research, multi-sensor fusion systems, and command-and-control testbeds where major components of the software stack execute outside the mobile device environment.

A representative example is the \texttt{atak\_bridge} node proposed by Larkin et al.\ \cite{Larkin2021ATAKIT}, which enables bidirectional communication between ATAK clients and autonomous air-ground robot teams. Implemented as a lightweight ROS node, \texttt{atak\_bridge} converts robot telemetry, object detections, and mission commands into CoT events, allowing robots to appear as geolocated entities within ATAK's Common Operating Picture. Standard ATAK tools, such as target designation and map-based point selection, can then be used to issue navigation goals without requiring any changes to the ATAK application.

This external bridging paradigm demonstrates how CoT can act as a unifying protocol between ATAK and heterogeneous autonomy systems. However, existing implementations, including \texttt{atak\_bridge} generally assume reliable Wi-Fi or wired network connectivity and continuous power availability. As a result, they are typically evaluated in laboratory settings or controlled field experiments and do not address communication in environments where infrastructure is absent, unstable, or energy-constrained. Moreover, because these adapters operate outside ATAK's plugin layer, they cannot leverage internal mechanisms for message persistence, contact management, or direct interaction with ATAK's communication pipeline.

These constraints motivate the design choice pursued in this thesis: a native ATAK plugin coupled to a LoRa-based communication layer. By integrating directly into ATAK's internal event flow and communication stack, the proposed approach provides sustained situational awareness even when conventional MANET, Wi-Fi, or IP-based communication substrates are unavailable.
