\subsection{Background}
Building decentralized communication systems that operate without infrastructure in disaster-stricken areas or tactical situations often depends on flexible and efficient physical-layer signal processing. However, conventional Software-Defined Radio (SDR) frameworks face several challenges in real-world deployments. They tend to lack adaptability across different platforms, offer limited real-time performance, and cannot easily be reconfigured at runtime. These limitations make them less suitable for applications that require low latency, dynamic control, and support for heterogeneous hardware environments.
GNU Radio, a widely used open-source SDR framework, illustrates these issues. In its design, each signal processing block, such as those for modulation, demodulation, or filtering, is assigned a dedicated thread. Communication between blocks occurs through shared buffers managed by a scheduler. While this architecture provides a clear and modular structure, the frequent thread switching introduces significant overhead. As a result, it is inefficient on embedded systems and lacks support for fine-grained scheduling or hardware acceleration across diverse computing platforms.
In contrast to traditional thread-bound models, FutureSDR, introduced by Volz et al., adopts a modular and task-based scheduling architecture. A centralized scheduler dynamically coordinates the execution of processing blocks based on their dependencies and system resource availability. Unlike frameworks that bind one thread to each block, FutureSDR allows a single thread to process multiple blocks in sequence. This reduces thread count, lowers context-switching costs, and improves compatibility with resource-constrained embedded platforms.
Moreover, FutureSDR supports cross-platform deployment through a backend abstraction mechanism. Signal processing components can seamlessly migrate across different hardware targets—including CPUs, GPUs, and FPGAs, depending on performance requirements or resource availability. This architectural flexibility enables efficient execution in heterogeneous environments and is particularly valuable in applications that demand both portability and low power consumption.

In multi-party, resource-limited environments such as disaster zones or tactical operations, a system capable of real-time information sharing is essential for enabling field personnel to make timely and informed decisions. The ATAK is an open-source software platform that runs on Android devices. Initially developed by the U.S. Department of Defense to support military operations, ATAK has since found broad adoption in civilian contexts.
In the United States, ATAK has been widely deployed across military units, special operations forces, search and rescue teams, and emergency management agencies. In Germany, the platform has also been adopted by law enforcement and disaster response teams, where users value its comprehensive feature set and modular design.
ATAK provides several key capabilities:
\begin{itemize}
    \item\textbf{Real-time geospatial collaboration:} Users can view both their own and others' positions on a shared map, mark areas of interest, and track movement in the field.

    \item\textbf{Data and message exchange:} The platform supports the sharing of layers, annotations, targets, tracks, and other mission-related data.

    \item\textbf{Plugin-based extensibility:} ATAK’s architecture allows for the integration of custom sensors, alternative communication interfaces, and external protocol parsers through modular plugins.
    
    \item\textbf{Cursor on Target (CoT) protocol:} This lightweight XML-based standard is used to represent events, positions, and status updates, serving as the core data format for ATAK’s internal communication.
\end{itemize}
 
The plugin architecture of ATAK makes it particularly well-suited for integrating the SDR-based communication systems proposed in this work. Its use of the CoT protocol and well-defined plugin APIs ensures that data exchange remains functional even in challenging environments, including scenarios without access to cellular networks or satellite links. By linking ATAK’s data layer to a physical-layer SDR interface, reliable communication can be maintained under extreme conditions.
