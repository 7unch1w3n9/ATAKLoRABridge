% !TeX root = ../Thesis.tex

%************************************************
\chapter{Introduction}\label{ch:introduction}
%************************************************
\glsresetall % Resets all acronyms to not used

\section{Motivation}

Currently, we live in an age defined by the interconnectedness of all things, where communication technologies evolve rapidly. 
Moreover, the advent of 5G technology signifies a pivotal breakthrough, providing faster data transmission, minimal delay, and extensive device connectivity. 
Such progress has greatly improved communication systems in multiple fields, particularly in areas like emergency rescue and disaster management. 
For rescue teams operating under time-sensitive and ever-changing conditions, their success largely relies on exchanging information promptly and ensuring the accuracy of data gathered at the scene.

\par

Nevertheless, the settings where these teams work frequently pose significant challenges to standard communication infrastructures. 
Major catastrophes, including earthquakes, tsunamis, volcanic activity, or deliberate damage to infrastructure, often cause severe destruction or disablement of communication networks. 
Consequently, communication blackouts occur, where even satellite systems struggle to maintain stable connections due to physical obstructions or excessive network demand. 
An illustrative case is the 2022 eruption of Hunga Tonga-Hunga Ha'apai, which cut undersea fiber-optic cables and hindered satellite signals because of volcanic ash, isolating Tonga from global communications for multiple days \cite{Tonga}.

\par

Such events emphasize the critical requirement for communication solutions that do not rely on permanent infrastructure. 
Specifically, communication technologies that are compact, energy-efficient, and support direct device-to-device interaction provide essential robustness in these conditions. 
By facilitating direct messaging between devices and decentralized information sharing, such technologies can preserve crucial communication channels during network failures.

\par
The primary objective of this paper is to address the aforementioned challenges by integrating and validating a prototype communication system. This system is designed to be infrastructure-independent and low-power, and its application within the Android Team Awareness Kit (ATAK), a mobile geospatial collaboration platform, will be thoroughly examined.

Specifically, the implementation leverages the open-source nature and plugin-based architecture of the ATAK platform to develop a dedicated communication plugin. This plugin features a loosely coupled design, enabling the seamless integration of existing physical-layer communication solutions, such as acoustic or LoRa technologies. Critically, the design allows for the flexible replacement of the underlying communication method based on operational requirements, all without necessitating modifications to the application-layer logic. This integration strategy ensures the system's capacity for reliable device-to-device communication even in dynamic and network-deprived environments.

The plugin-based integration scheme offers significant advantages, making the system particularly well-suited for disaster relief operations where communication infrastructure has suffered severe damage. By utilizing the powerful map visualization capabilities of the ATAK platform, all communication information and device locations can be presented in a unified and intuitive manner. This capability substantially enhances the situational awareness and collaborative efficiency of rescue teams. Furthermore, the proposed solution demonstrates broad versatility, with its applicability extending to specialized missions where conventional communication is often limited. Examples of such scenarios include firefighting, scientific expeditions, and remote medical support. 



