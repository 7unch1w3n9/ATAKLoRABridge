\section{Related Work}

In the specific context of disaster communication, Deepak G. and his collaborators, through their investigative research, introduced a classification methodology for scenarios based on the degree of damage sustained by communication networks. This classification encompasses congested networks, partially functional networks, and completely isolated networks\cite{Post-DisasterEmergencyCommunicationSystems}.

A congested network describes a situation where the communication infrastructure remains operational; however, its performance significantly degrades due to a surge in communication demands. In such instances, mechanisms like enhanced Mobile Priority Services (eMPS) or network slicing can be implemented to ensure the prioritization of emergency communications. A partially functional network signifies that certain nodes within the communication infrastructure have failed, resulting in connectivity being maintained only in localized areas or for specific services. Conversely, a completely isolated network represents the most extreme scenario, characterized by the failure of all base stations and a complete loss of communication connectivity within the affected area. Under the conditions of partially functional or completely isolated networks, traditional centralized communication systems often prove ineffective, thereby necessitating autonomous and decentralized communication solutions.

In recent years, the field of emergency communication has witnessed the emergence of various technologies specifically suited for scenarios involving infrastructure failure\cite{Post-DisasterEmergencyCommunicationSystems}. These include:
    \begin{itemize}
        \item \textbf{Device-to-Device communication}.\ This enables direct communication between end devices without requiring support from base stations.
        \item \textbf{Unmanned Aerial Vehicle relays}.\ These systems utilize UAVs equipped with communication devices to temporarily restore communication coverage in disaster-affected regions.
        \item \textbf{Mobile Ad-hoc Networks (MANETs)}. These networks are formed by mobile nodes that dynamically self-organize into network topologies, providing rapid and flexible wireless communication.
    \end{itemize}
    
Although D2D\footnote{Device-to-Device},  UAVs\footnote{Unmanned Aerial Vehicles}, and MANETs\footnote{Mobile Ad-hoc Networks } provide different forms of decentralized communication capabilities for post-disaster areas, they remain subject to significant limitations in terms of communication range, operational duration, and scalability in high-density environments. D2D typically covers only tens to hundreds of meters \cite{TODO}. MANET relies on Wi-Fi/short-range links, which are characterized by high energy consumption and bulky device form factors \cite{TODO}. UAVs are constrained by battery capacity and flight time\cite{TODO}. None of these three approaches can sustainably provide low-power connectivity across large-scale areas that are completely isolated with destroyed infrastructure. Consequently, an increasing body of research has recently focused on LPWAN\footnote{Low-Power Wide-Area Network} technologies characterized by ultra-low power consumption and long-range communication, particularly LoRa as candidate physical layer solutions for emergency communications.

\section{LoRa-based emergency communication systems}


Building on the physical-layer characteristics of LoRa summarized in Section 1.3.3, a number of application-layer systems have been proposed that exploit its long-range, low-power links for use in disrupted or infrastructure-less environments. Among these, two representative designs are LoRAgent and Meshtastic, which illustrate complementary approaches to integrating LoRa into higher-layer communication architectures for emergency response.

\subsection{LoRAgent}
LoRAgent is a delay-tolerant networking (DTN) solution that leverages LoRa links to facilitate message delivery in challenged environments\cite{Putrenko2024}. 

The LoRAgent system comprises two types of nodes. One type consists of fixedly deployed relay nodes responsible for caching and forwarding messages. The other type involves mobile nodes that "carry" information in areas without network coverage until they encounter a suitable forwarding target, enabling non-real-time message delivery. To enhance forwarding efficiency, LoRAgent incorporates a quadrant-based directional forwarding strategy and a historical record filtering mechanism. This reduces redundant relays and improves transmission efficacy.

\subsection{Meshtastic}

In contrast to LoRAgent's delay-tolerant architecture, Meshtastic offers a LoRa-based solution focused on real-time communication\cite{AboutMeshtastic}. It is built upon a self-organizing mesh network architecture, achieving multi-hop message forwarding through continuous broadcasting and relaying. By employing an event-driven transmission mode and a lightweight message structure, Meshtastic maintains both low power consumption and minimal latency. This makes it particularly well-suited for deployment on handheld or small-scale devices. In practical applications, developers have successfully integrated Meshtastic into ATAK through a plugin\cite{meshtasticATAKPlugin}.

Overall, existing ATAK-LoRa integrations—such as Meshtastic-based plugins—demonstrate the feasibility of using LoRa as a low-power communication channel for field deployments. These solutions, however, are architecturally tied to LoRa as a fixed physical layer, and therefore offer limited flexibility when alternative communication technologies (e.g., acoustic modems, sub-GHz FSK, BLE long-range, or other SDR-based waveforms) may be required in future scenarios. This highlights an open opportunity for systems that maintain the lightweight, infrastructure-independent characteristics of LoRa-style transport while remaining adaptable to evolving or heterogeneous physical layers. The subsequent chapters build on this observation.

\subsection{FutureSDR and Java Native Interface (JNI)}

    However, systems such as LoRAgent and Meshtastic typically rely on proprietary communication stacks or dedicated hardware chipsets, which constrains flexibility and limits deeper customization. To fully leverage LoRa's capabilities and integrate it seamlessly into a wide range of mobile devices, a portable, efficient, and open-source Software -Defined Radio (SDR) framework for underlying signal processing is essential.

    Traditional SDR frameworks face challenges in platform adaptability, real-time performance, and runtime configuration, often falling short of the low-latency, cross-platform, and dynamically controllable requirements of real-world deployments. For instance, GNU Radio assigns each signal processing block, such as those for modulation, demodulation, and filtering, a dedicated thread and relies on buffered inter-thread communication for data transfer. While this model offers clarity and modularity, it incurs substantial thread-switching overhead, lacks support for fine-grained scheduling or heterogeneous hardware, and is generally ill-suited for embedded systems.

    In contrast, FutureSDR, introduced by Volz et al., employs a modular scheduling architecture that eliminates the resource overhead associated with thread-per-block designs. A centralized scheduler coordinates task execution based on data dependencies and resource availability, allowing a single thread to execute multiple blocks in sequence. This structure significantly reduces threading overhead and improves suitability for resource-constrained embedded devices. Moreover, FutureSDR incorporates a backend abstraction layer that enables signal processing modules to run across diverse computing platforms, including general-purpose CPUs, GPUs, and FPGAs, thereby delivering high portability and execution efficiency in heterogeneous or constrained environments.

    To integrate the Rust-based LoRa signal processing library, which is built on FutureSDR, into the ATAK plugin environment, a robust bridge between the Java application layer and native code is required. The Java Native Interface (JNI) serves as an ideal solution, being the standard native programming interface for the Java platform. A key strength of JNI is its binary compatibility: a compiled native library can execute consistently across different Java Virtual Machine implementations without platform-specific recompilation\cite{TODO}. This feature is particularly valuable for ATAK plugins targeting diverse Android devices.

    Architecturally, JNI uses an interface pointer table, which is a pointer to an array of function pointers, to access native methods. This design ensures a stable interface while accommodating varying VM implementations. In practice, Java methods are declared with the native keyword, with their implementations residing in dynamic shared libraries such as .so files. The VM resolves these symbols at runtime using a structured name-mapping convention.

    JNI also provides a complete type-mapping mechanism. Primitive Java types, for example int and double, are mapped to native equivalents such as jint and jdouble, while Java objects are handled via the jobject hierarchy. Memory management combines both local and global references. Local references are automatically reclaimed after native method execution, making them ideal for short-lived objects. Global references persist until explicitly released and are suitable for long-lived cross-context references. This dual model balances ease of use with safe memory practices.

    Additionally, JNI supports full exception handling, allowing native code to catch, check, or throw Java exceptions. The specification also enforces thread locality of JNI interface pointers, preventing their use across threads and ensuring thread-safe operation in concurrent environments\cite{TODO}. Together, these characteristics establish JNI as a reliable and efficient bridge between Java applications and native libraries, forming a solid technical basis for integrating ATAK plugins with FutureSDR flowgraphs.
\subsection{ATAK}
The integration path of Meshtastic as a plugin within the Android Team Awareness Kit demonstrates the feasibility of embedding infra\-structure-independent communication modules into a robust situational awareness platform. Among various platforms, ATAK stands out as an ideal integration target due to its highly modular architecture, real-world validated stability, and extensive support for plugins. Originally developed by the United States Department of Defense to fulfill military tactical requirements, the platform has subsequently garnered increasing popularity within civilian sectors\cite{ATAK}. Its widespread deployment across military units, special police forces, search and rescue teams, and emergency management agencies in the United States exemplifies its utility. In Germany, smartphone-based applications designed for emergency alerting have become increasingly vital tools for first responders, including personnel from fire departments, police forces, and various rescue organizations\cite{Linde2022EmergencyAT}. This proliferation aligns with Germany's decentralized model of emergency and disaster management. Within this framework, responsibilities are collectively shared across federal, state, and local governmental tiers, fostering collaborative participation between official agencies and non-governmental organizations\cite{Domres2000TheGA}. In this context, the functionalities offered by ATAK are highly congruent with the German emergency response system's demand for adaptable and efficient information processing instruments. This synergy indicates a substantial potential for its broader application within the German operational landscape.

To accommodate the complex and dynamic demands of tactical communication, ATAK's architectural design incorporates a highly modular approach, supporting the rapid integration of multi-source information and communication interfaces through its plugin mechanism.  Its user interface is centered around a map-based design, supporting the integration of multiple online map sources through the MOBAC\footnote{Mobile Atlas Creator} format. The platform also offers offline caching capabilities, allowing users to access high-resolution geospatial data even in disconnected environments. The overlay feature enables the display of static images, such as heat maps or scanned paper maps, directly on top of the digital map view, further enhancing tactical planning and situational awareness. In addition, ATAK provides a comprehensive set of annotation and collaboration tools: users can place standardized icons, draw, and annotate on the map, and share these markings in real time via team-wide broadcast mechanisms. The inherent adaptability of the platform's communication layer allows it to flexibly accommodate heterogeneous physical networks, including LTE, Wi-Fi, MANETs, and satellite links. In extreme scenarios where central routers or trusted nodes are absent, ATAK's built-in GeoChat module can establish a logical broadcast domain using UDP multicast. This facilitates local information synchronization through a peer-to-peer approach. GeoChat operates without relying on any infrastructure, making it suitable for rapid deployment and use in low-resource environments. However, a limitation of this module is its lack of message persistence. If a recipient is offline when a message is sent, the content won't be delivered, and there's no mechanism for retransmission. To address the need for persistent communication, the platform incorporates the TAKChat plugin, which is built on the XMPP protocol. This module provides full chat persistence, supporting cross-platform message delivery, server-side storage, and chat room permission management.

All data is ultimately encapsulated within the Cursor-on-Target (CoT) protocol, a lightweight XML-based structure developed by MITRE\cite{Jacoby2014AfterAR}. This protocol features a clear event model and semantic hierarchy. Each CoT message is composed of an \texttt{<event>} element, which includes fields such as a unique identifier (\texttt{uid}), event type (\texttt{type}), time information (\texttt{start}, \texttt{stale}, \texttt{time}), generation method (\texttt{how}), geographical coordinates (\texttt{<point>}), and extended details (\texttt{<detail>}). The system utilizes the \texttt{\textless detail\textgreater} node to flexibly carry extended data, such as \texttt{\_\_chat} for message bodies, \texttt{chatgrp} for group identifiers, \texttt{link} for associating entity relationships,  and \\\texttt{\_\_serverdestination} for explicitly specifying the message's target address and transmission protocol (e.g., \texttt{IP:PORT:PROTOCOL}). This protocol format is lightweight and semantically clear, supporting various information types including event types, geographical locations, and message content, thereby forming the foundation for inter-plugin collaboration and system compatibility.

\subsubsection{ATAK's Plugin Architecture for Customization}
The plugin mechanism of ATAK is one of its most distinctive architectural features. Plugins exist as independent \texttt{APK} (Android Package Kit) files and do not rely on a whitelist from the main application. Instead, the main application automatically identifies, loads, and registers them at runtime. Plugins extend modular functionality based on predefined TAK APIs, gaining direct access to core functional modules such as map rendering, message distribution, databases, location information, and contact lists.

Typical class structures within this architecture include:

\begin{itemize}
  \item \textbf{PluginLifecycle}: This component functions as the plugin's life cycle manager, analogous to Android's \texttt{Application} class, and is responsible for initialization and configuration hooks.

  \item \textbf{DropDownMapComponent}: This acts as the main controller, registering user interface logic, event listeners, and task scheduling components. It is functionally similar to an Android \texttt{Activity}.

  \item \textbf{DropDownReceiver}: This is primarily responsible for UI display, typically bound to a slide-out panel, and supports loading custom XML layouts.
\end{itemize}

The method DropDownReceiver.showDropDown() is responsible for presenting user interface elements in response to user actions such as button clicks, map interactions, or background events.


\vspace{0.5em}
Developing ATAK plugins requires Android Studio, with a minimum supported Android version of 5.0 (\texttt{API 21}) or higher. It is recommended to use Java 11, Gradle Plugin version \texttt{4.2.2}, and main project Gradle version \texttt{6.9.1} to ensure compatibility with the existing \texttt{ATAK-CIV} SDK. Plugins must possess a signing key for system validation, and their deployment path should be located within the \texttt{atak-civ/plugins} directory, maintaining consistent directory hierarchy with components such as \texttt{main.jar} and \texttt{atak.apk} within the SDK to ensure automatic loading.


