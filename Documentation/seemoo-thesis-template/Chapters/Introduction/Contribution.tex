\section{Contribution}

The primary objective of this paper is to address the aforementioned challenges by integrating and validating a prototype communication system. This system is designed to be infrastructure-independent and low-power, and its application within the Android Team Awareness Kit (ATAK), a mobile geospatial collaboration platform, will be thoroughly examined.


Specifically, the implementation leverages the open-source nature and plugin-based architecture of the ATAK platform to develop a dedicated communication plugin. This plugin features a loosely coupled design, enabling the seamless integration of existing physical-layer communication solutions, such as acoustic or LoRa technologies. Critically, the design allows for the flexible replacement of the underlying communication method based on operational requirements, all without necessitating modifications to the application-layer logic. This integration strategy ensures the system's capacity for reliable device-to-device communication even in dynamic and 
network-deprived environments.


The plugin-based integration scheme offers significant advantages, making the system particularly well-suited for disaster relief operations where communication infrastructure has suffered severe damage. By utilizing the powerful map visualization capabilities of the ATAK platform, all communication information and device locations can be presented in a unified and intuitive manner. This capability substantially enhances the situational awareness and collaborative efficiency of rescue teams. Furthermore, the proposed solution demonstrates broad versatility, with its applicability extending to specialized missions where conventional communication is often limited. Examples of such scenarios include firefighting, scientific expeditions, and remote medical support. 
