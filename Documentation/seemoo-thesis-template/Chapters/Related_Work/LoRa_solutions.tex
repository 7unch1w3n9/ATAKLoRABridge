\subsection{LoRAgent and Meshtastic}
LoRa technology offers significant advantages over traditional communication methods like D2D\footnote{Device-to-Device},  UAVs\footnote{Unmanned Aerial Vehicles}, and MANETs\footnote{Mobile Ad-hoc Networks } in disaster communication scenarios. Its notable benefits include long-range communication, low power consumption, and strong interference immunity. Numerous LoRa-based communication solutions have been proposed, with LoRAgent standing out as a system architecture that integrates geographical awareness with delay-tolerant mechanisms \cite{LoRAgent}.

The LoRAgent system comprises two types of nodes. One type consists of fixedly deployed relay nodes responsible for caching and forwarding messages. The other type involves mobile nodes that "carry" information in areas without network coverage until they encounter a suitable forwarding target, enabling non-real-time message delivery. To enhance forwarding efficiency, LoRAgent incorporates a quadrant-based directional forwarding strategy and a historical record filtering mechanism. This reduces redundant relays and improves transmission efficacy.

In contrast to LoRAgent's delay-tolerant architecture, Meshtastic offers a LoRa-based solution focused on real-time communication\cite{AboutMeshtastic}. It is built upon a self-organizing mesh network architecture, achieving multi-hop message forwarding through continuous broadcasting and relaying. By employing an event-driven transmission mode and a lightweight message structure, Meshtastic maintains both low power consumption and minimal latency. This makes it particularly well-suited for deployment on handheld or small-scale devices. In practical applications, developers have successfully integrated Meshtastic into ATAK through a plugin\cite{meshtasticATAKPlugin}.