\subsection{LoRa, LoRAgent and Meshtastic}

LoRa is a representative Low-Power Wide-Area (LPWA) wireless communication technology whose physical layer is based on the proprietary Chirp Spread Spectrum (CSS) modulation scheme proposed by Semtech. CSS encodes information by linearly sweeping the carrier frequency, and its spread spectrum characteristics endow LoRa with inherent robustness against narrowband interference, multipath fading, and Doppler effects. Consequently, LoRa can achieve reliable communication over distances of several kilometers with extremely low transmission power in sub-GHz ISM bands such as 433/868/915 MHz. The performance of LoRa links is primarily determined by the spreading factor (SF7–SF12), channel bandwidth (125–500 kHz), and coding rate (CR 4/5–4/8). Increasing the spreading factor extends the symbol duration, enabling the receiver to successfully demodulate signals at extremely low signal-to-noise ratios down to -20 dB, at the cost of reduced data rate. The coding rate enhances interference immunity through forward error correction, while the bandwidth determines the chip rate, thereby affecting sensitivity and symbol duration. A key characteristic of LoRa is the near-orthogonality between different spreading factors: frames using different SFs are distinguished during the correlation-based demodulation process at the receiver, and even when transmitted simultaneously on the same frequency, they exhibit minimal mutual interference. This mechanism enables LoRa to maintain high scalability and reliability within limited spectrum resources. Combined with its link budget of up to 150-157 dB and idle current consumption in the microampere range, LoRa can operate for several years under battery-powered conditions. This unique combination of long range, strong robustness, and ultra-low power consumption establishes the technical foundation for its application in disaster emergency communication scenarios characterized by infrastructure failure and energy constraints.

These technical merits have spurred the development of various application-layer systems that leverage LoRa for emergency communications. Among the numerous LoRa-based solutions proposed, LoRAgent stands out as a system architecture that integrates geographical awareness with delay-tolerant mechanisms\cite{LoRAgent}.

The LoRAgent system comprises two types of nodes. One type consists of fixedly deployed relay nodes responsible for caching and forwarding messages. The other type involves mobile nodes that "carry" information in areas without network coverage until they encounter a suitable forwarding target, enabling non-real-time message delivery. To enhance forwarding efficiency, LoRAgent incorporates a quadrant-based directional forwarding strategy and a historical record filtering mechanism. This reduces redundant relays and improves transmission efficacy.

In contrast to LoRAgent's delay-tolerant architecture, Meshtastic offers a LoRa-based solution focused on real-time communication\cite{AboutMeshtastic}. It is built upon a self-organizing mesh network architecture, achieving multi-hop message forwarding through continuous broadcasting and relaying. By employing an event-driven transmission mode and a lightweight message structure, Meshtastic maintains both low power consumption and minimal latency. This makes it particularly well-suited for deployment on handheld or small-scale devices. In practical applications, developers have successfully integrated Meshtastic into ATAK through a plugin\cite{meshtasticATAKPlugin}.