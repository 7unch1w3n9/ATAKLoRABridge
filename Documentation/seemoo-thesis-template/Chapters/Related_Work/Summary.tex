\section{Summary}
Based on the preceding analysis, we provide a consolidated overview of the existing
literature as follows:

We first examined the state of the art in situational awareness (SA) platforms and
low-power communication technologies. This included an analysis of both general and
operational SA systems, such as Delta and ATAK, acknowledging their advanced
collaboration capabilities while highlighting their reliance on high-bandwidth
communication infrastructures. We then reviewed ATAK's plugin architecture and its
existing external CoT-based integration approaches, outlining their strengths as well as
their limitations in terms of extensibility and interoperability.

Subsequently, we turned to low-power wide-area communication solutions and assessed
LoRa-based systems such as LoRAgent and Meshtastic. These works demonstrate the
practicality of long-range, low-power communication in constrained environments,
though they typically remain separate from mainstream situational awareness
ecosystems. In addition, we briefly revisited Software-Defined Radio (SDR) frameworks
and the Java Native Interface (JNI), which provide essential building blocks for portable
signal processing.

Overall, while existing solutions each offer distinct advantages, they collectively reveal
an open space that has not yet been adequately addressed: current systems tend to be
tied to specific physical-layer implementations and lack deeper integration with the
internal data flows and event models of situational awareness platforms. Addressing this
open space forms the basis for the research explored in the chapters that follow.
