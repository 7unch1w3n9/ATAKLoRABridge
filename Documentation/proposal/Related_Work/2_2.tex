\subsection{LoRa-Based Infrastructure-Free Communication Solutions}
To overcome the inherent limitations of traditional D2D, MANET and UAV relay methods in disaster communication, Baumgärtner et al. introduced an innovative emergency communication system known as LoRAgent. This system uniquely integrates LoRa technology with Delay-Tolerant Networks (DTN). This synergistic integration capitalizes on LoRa's intrinsic advantages, including low power consumption, extended transmission range, and strong signal penetration capabilities through obstacles. Concurrently, it facilitates information exchange even in environments completely lacking conventional communication infrastructure.

The LoRAgent system is designed with two distinct categories of nodes:
\begin{itemize}
    \item\textbf{relay nodes} are typically built on Raspberry Pi platforms equipped with both LoRa and GPS modules. Their primary function is to buffer and forward messages, acting as key intermediaries within the communication infrastructure.

    \item\textbf{pager devices} are designed to directly receive alerts via LoRa or connect to mobile phones through Bluetooth, thereby extending communication capabilities and enabling user interaction.
\end{itemize}

The operational principle of the system is based on a quintessential "store-carry-forward" model. This model enables mobile nodes to transport messages until an appropriate forwarding destination is encountered. Unlike most conventional LoRa communication schemes, which primarily depend on static identifiers for routing or simple broadcasting without considering geographical location, LoRAgent incorporates a sophisticated quadrant-based geographical forwarding mechanism. This mechanism enhances the efficiency of directional message propagation by prioritizing neighboring nodes and utilizing a forwarding history table. This history table serves to prevent redundant retransmissions of messages to nodes that have already received them.

In comparison, Meshtastic represents another significant LoRa-based communication solution, though its design philosophy places a greater emphasis on real-time communication. It is built upon a self-organizing mesh network architecture, achieving multi-hop message forwarding through continuous broadcasting and relaying. By employing an event-driven transmission mode and a lightweight message structure, Meshtastic maintains both low power consumption and minimal latency. This makes it particularly well-suited for deployment on handheld or or small-scale devices. In practical applications, developers have successfully integrated Meshtastic into ATAK through a plugin. This integration allows for location sharing and basic message transmission via LoRa nodes, even in situations where public network access or satellite links are unavailable.