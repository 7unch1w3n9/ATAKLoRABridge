\section{Motivation}

Currently, we live in an age defined by the interconnectedness of all things, where communication technologies evolve rapidly. 
Moreover, the advent of 5G technology signifies a pivotal breakthrough, providing faster data transmission, minimal delay, and extensive device connectivity. 
Such progress has greatly improved communication systems in multiple fields, particularly in areas like emergency rescue and disaster management. 
For rescue teams operating under time-sensitive and ever-changing conditions, their success largely relies on exchanging information promptly and ensuring the accuracy of data gathered at the scene.

\par

Nevertheless, the settings where these teams work frequently pose significant challenges to standard communication infrastructures. 
Major catastrophes, including earthquakes, tsunamis, volcanic activity, or deliberate damage to infrastructure, often cause severe destruction or disablement of communication networks. 
Consequently, communication blackouts occur, where even satellite systems struggle to maintain stable connections due to physical obstructions or excessive network demand. 
An illustrative case is the 2022 eruption of Hunga Tonga-Hunga Ha'apai, which cut undersea fiber-optic cables and hindered satellite signals because of volcanic ash, isolating Tonga from global communications for multiple days.

\par

Such events emphasize the critical requirement for communication solutions that do not rely on permanent infrastructure. 
Specifically, communication technologies that are compact, energy-efficient, and support direct device-to-device interaction provide essential robustness in these conditions. 
By facilitating direct messaging between devices and decentralized information sharing, such technologies can preserve crucial communication channels during network failures.

\par

This thesis aims to demonstrate and validate the integration of an infrastructure-independent, low-power communication prototype into the mobile geospatial coordination platform ATAK\footnote{Android Team Awareness Kit}. 
The system enables point-to-point or multipoint data exchange via LoRa technologies in the absence of conventional network infrastructure, offering high adaptability to various environmental conditions. 
While its advantages are particularly significant in disaster response scenarios, the system also shows potential for broader application in routine communication-limited contexts, such as firefighting missions, scientific expeditions, and remote medical support. 
By leveraging ATAK's powerful visualization capabilities, all transmitted communication and location data can be displayed on a unified map interface, thereby enhancing situational awareness and coordination efficiency during missions.