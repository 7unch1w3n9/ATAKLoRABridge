\section{LoRa-based emergency communication systems}


Building on the physical-layer characteristics of LoRa summarized in Section 1.3.3, a number of application-layer systems have been proposed that exploit its long-range, low-power links for use in disrupted or infrastructure-less environments. Among these, two representative designs are LoRAgent and Meshtastic, which illustrate complementary approaches to integrating LoRa into higher-layer communication architectures for emergency response.

\subsection{LoRAgent}
LoRAgent is a delay-tolerant networking (DTN) solution that leverages LoRa links to facilitate message delivery in challenged environments\cite{Putrenko2024}. 

The LoRAgent system comprises two types of nodes. One type consists of fixedly deployed relay nodes responsible for caching and forwarding messages. The other type involves mobile nodes that "carry" information in areas without network coverage until they encounter a suitable forwarding target, enabling non-real-time message delivery. To enhance forwarding efficiency, LoRAgent incorporates a quadrant-based directional forwarding strategy and a historical record filtering mechanism. This reduces redundant relays and improves transmission efficacy.

\subsection{Meshtastic}

In contrast to LoRAgent's delay-tolerant architecture, Meshtastic offers a LoRa-based solution focused on real-time communication\cite{AboutMeshtastic}. It is built upon a self-organizing mesh network architecture, achieving multi-hop message forwarding through continuous broadcasting and relaying. By employing an event-driven transmission mode and a lightweight message structure, Meshtastic maintains both low power consumption and minimal latency. This makes it particularly well-suited for deployment on handheld or small-scale devices. In practical applications, developers have successfully integrated Meshtastic into ATAK through a plugin\cite{meshtasticATAKPlugin}.

Overall, existing ATAK-LoRa integrations—such as Meshtastic-based plugins—demonstrate the feasibility of using LoRa as a low-power communication channel for field deployments. These solutions, however, are architecturally tied to LoRa as a fixed physical layer, and therefore offer limited flexibility when alternative communication technologies (e.g., acoustic modems, sub-GHz FSK, BLE long-range, or other SDR-based waveforms) may be required in future scenarios. This highlights an open opportunity for systems that maintain the lightweight, infrastructure-independent characteristics of LoRa-style transport while remaining adaptable to evolving or heterogeneous physical layers. The subsequent chapters build on this observation.
