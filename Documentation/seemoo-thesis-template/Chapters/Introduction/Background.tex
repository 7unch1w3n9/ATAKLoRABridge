\section{Background}

To situate this work within the broader field of emergency communication, it is essential to first understand how communication networks behave under disaster conditions.

\subsection{Disaster Communication Models}

In the specific context of disaster communication, Deepak G. and his collaborators, through their investigative research, introduced a classification methodology for scenarios based on the degree of damage sustained by communication networks. This classification encompasses congested networks, partially functional networks, and completely isolated networks\cite{Post-DisasterEmergencyCommunicationSystems}.

A congested network describes a situation where the communication infrastructure remains operational; however, its performance significantly degrades due to a surge in communication demands. In such instances, mechanisms like enhanced Mobile Priority Services (eMPS) or network slicing can be implemented to ensure the prioritization of emergency communications. A partially functional network signifies that certain nodes within the communication infrastructure have failed, resulting in connectivity being maintained only in localized areas or for specific services. Conversely, a completely isolated network represents the most extreme scenario, characterized by the failure of all base stations and a complete loss of communication connectivity within the affected area. Under the conditions of partially functional or completely isolated networks, traditional centralized communication systems often prove ineffective, thereby necessitating autonomous and decentralized communication solutions.

Understanding these failure models is essential for evaluating any disaster-oriented communication solution, as different levels of infrastructure degradation impose distinct requirements on range, energy consumption, and network
independence.


\subsection{Decentralized Communication Technologies}

In recent years, the field of emergency communication has witnessed the emergence of various technologies specifically suited for scenarios involving infrastructure failure\cite{Post-DisasterEmergencyCommunicationSystems}. These include:
    \begin{itemize}
        \item \textbf{Device-to-Device communication}.\ This enables direct communication between end devices without requiring support from base stations.
        \item \textbf{Unmanned Aerial Vehicle relays}.\ These systems utilize UAVs equipped with communication devices to temporarily restore communication coverage in disaster-affected regions.
        \item \textbf{Mobile Ad-hoc Networks (MANETs)}. These networks are formed by mobile nodes that dynamically self-organize into network topologies, providing rapid and flexible wireless communication.
    \end{itemize}
    
Although D2D\footnote{Device-to-Device},  UAVs\footnote{Unmanned Aerial Vehicles}, and MANETs\footnote{Mobile Ad-hoc Networks } provide different forms of decentralized communication capabilities for post-disaster areas, they remain subject to significant limitations in terms of communication range, operational duration, and scalability in high-density environments. D2D typically covers only tens to hundreds of meters \cite{TODO}. MANET relies on Wi-Fi/short-range links, which are characterized by high energy consumption and bulky device form factors \cite{TODO}. UAVs are constrained by battery capacity and flight time\cite{TODO}. None of these three approaches can sustainably provide low-power connectivity across large-scale areas that are completely isolated with destroyed infrastructure. Consequently, an increasing body of research has recently focused on LPWAN\footnote{Low-Power Wide-Area Network} technologies characterized by ultra-low power consumption and long-range communication, particularly LoRa as candidate physical layer solutions for emergency communications.

\subsection{LoRa}

LoRa is a representative Low-Power Wide-Area (LPWA) wireless communication technology whose physical layer is based on the proprietary Chirp Spread Spectrum (CSS) modulation scheme proposed by Semtech. CSS encodes information by linearly sweeping the carrier frequency, and its spread spectrum characteristics endow LoRa with inherent robustness against narrowband interference, multipath fading, and Doppler effects. Consequently, LoRa can achieve reliable communication over distances of several kilometers with extremely low transmission power in sub-GHz ISM bands such as 433, 868, and 915 MHz. 

The performance of LoRa links is primarily determined by the spreading factor (SF7–SF12), channel bandwidth (125–500 kHz), and coding rate (CR 4/5–4/8). Increasing the spreading factor extends the symbol duration, enabling the receiver to successfully demodulate signals at extremely low signal-to-noise ratios down to -20 dB, at the cost of reduced data rate. The coding rate enhances interference immunity through forward error correction, while the bandwidth determines the chip rate, thereby affecting sensitivity and symbol duration. A key characteristic of LoRa is the near-orthogonality between different spreading factors: frames using different SFs are distinguished during the correlation-based demodulation process at the receiver, and even when transmitted simultaneously on the same frequency, they exhibit minimal mutual interference. This mechanism enables LoRa to maintain high scalability and reliability within limited spectrum resources. 

Combined with its link budget of up to 150-157 dB and idle current consumption in the microampere range, LoRa can operate for several years under battery-powered conditions. This unique combination of long range, strong robustness, and ultra-low power consumption establishes the technical foundation for its application in disaster emergency communication scenarios characterized by infrastructure failure and energy constraints.

